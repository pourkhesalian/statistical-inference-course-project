\documentclass[]{article}
\usepackage{lmodern}
\usepackage{amssymb,amsmath}
\usepackage{ifxetex,ifluatex}
\usepackage{fixltx2e} % provides \textsubscript
\ifnum 0\ifxetex 1\fi\ifluatex 1\fi=0 % if pdftex
  \usepackage[T1]{fontenc}
  \usepackage[utf8]{inputenc}
\else % if luatex or xelatex
  \ifxetex
    \usepackage{mathspec}
  \else
    \usepackage{fontspec}
  \fi
  \defaultfontfeatures{Ligatures=TeX,Scale=MatchLowercase}
\fi
% use upquote if available, for straight quotes in verbatim environments
\IfFileExists{upquote.sty}{\usepackage{upquote}}{}
% use microtype if available
\IfFileExists{microtype.sty}{%
\usepackage{microtype}
\UseMicrotypeSet[protrusion]{basicmath} % disable protrusion for tt fonts
}{}
\usepackage[margin=1in]{geometry}
\usepackage{hyperref}
\hypersetup{unicode=true,
            pdftitle={Moment of Truth: Does Central Limit Theorem Work on an Exponential Distribution?},
            pdfauthor={Ali Pourkhesalian},
            pdfborder={0 0 0},
            breaklinks=true}
\urlstyle{same}  % don't use monospace font for urls
\usepackage{color}
\usepackage{fancyvrb}
\newcommand{\VerbBar}{|}
\newcommand{\VERB}{\Verb[commandchars=\\\{\}]}
\DefineVerbatimEnvironment{Highlighting}{Verbatim}{commandchars=\\\{\}}
% Add ',fontsize=\small' for more characters per line
\usepackage{framed}
\definecolor{shadecolor}{RGB}{248,248,248}
\newenvironment{Shaded}{\begin{snugshade}}{\end{snugshade}}
\newcommand{\AlertTok}[1]{\textcolor[rgb]{0.94,0.16,0.16}{#1}}
\newcommand{\AnnotationTok}[1]{\textcolor[rgb]{0.56,0.35,0.01}{\textbf{\textit{#1}}}}
\newcommand{\AttributeTok}[1]{\textcolor[rgb]{0.77,0.63,0.00}{#1}}
\newcommand{\BaseNTok}[1]{\textcolor[rgb]{0.00,0.00,0.81}{#1}}
\newcommand{\BuiltInTok}[1]{#1}
\newcommand{\CharTok}[1]{\textcolor[rgb]{0.31,0.60,0.02}{#1}}
\newcommand{\CommentTok}[1]{\textcolor[rgb]{0.56,0.35,0.01}{\textit{#1}}}
\newcommand{\CommentVarTok}[1]{\textcolor[rgb]{0.56,0.35,0.01}{\textbf{\textit{#1}}}}
\newcommand{\ConstantTok}[1]{\textcolor[rgb]{0.00,0.00,0.00}{#1}}
\newcommand{\ControlFlowTok}[1]{\textcolor[rgb]{0.13,0.29,0.53}{\textbf{#1}}}
\newcommand{\DataTypeTok}[1]{\textcolor[rgb]{0.13,0.29,0.53}{#1}}
\newcommand{\DecValTok}[1]{\textcolor[rgb]{0.00,0.00,0.81}{#1}}
\newcommand{\DocumentationTok}[1]{\textcolor[rgb]{0.56,0.35,0.01}{\textbf{\textit{#1}}}}
\newcommand{\ErrorTok}[1]{\textcolor[rgb]{0.64,0.00,0.00}{\textbf{#1}}}
\newcommand{\ExtensionTok}[1]{#1}
\newcommand{\FloatTok}[1]{\textcolor[rgb]{0.00,0.00,0.81}{#1}}
\newcommand{\FunctionTok}[1]{\textcolor[rgb]{0.00,0.00,0.00}{#1}}
\newcommand{\ImportTok}[1]{#1}
\newcommand{\InformationTok}[1]{\textcolor[rgb]{0.56,0.35,0.01}{\textbf{\textit{#1}}}}
\newcommand{\KeywordTok}[1]{\textcolor[rgb]{0.13,0.29,0.53}{\textbf{#1}}}
\newcommand{\NormalTok}[1]{#1}
\newcommand{\OperatorTok}[1]{\textcolor[rgb]{0.81,0.36,0.00}{\textbf{#1}}}
\newcommand{\OtherTok}[1]{\textcolor[rgb]{0.56,0.35,0.01}{#1}}
\newcommand{\PreprocessorTok}[1]{\textcolor[rgb]{0.56,0.35,0.01}{\textit{#1}}}
\newcommand{\RegionMarkerTok}[1]{#1}
\newcommand{\SpecialCharTok}[1]{\textcolor[rgb]{0.00,0.00,0.00}{#1}}
\newcommand{\SpecialStringTok}[1]{\textcolor[rgb]{0.31,0.60,0.02}{#1}}
\newcommand{\StringTok}[1]{\textcolor[rgb]{0.31,0.60,0.02}{#1}}
\newcommand{\VariableTok}[1]{\textcolor[rgb]{0.00,0.00,0.00}{#1}}
\newcommand{\VerbatimStringTok}[1]{\textcolor[rgb]{0.31,0.60,0.02}{#1}}
\newcommand{\WarningTok}[1]{\textcolor[rgb]{0.56,0.35,0.01}{\textbf{\textit{#1}}}}
\usepackage{graphicx,grffile}
\makeatletter
\def\maxwidth{\ifdim\Gin@nat@width>\linewidth\linewidth\else\Gin@nat@width\fi}
\def\maxheight{\ifdim\Gin@nat@height>\textheight\textheight\else\Gin@nat@height\fi}
\makeatother
% Scale images if necessary, so that they will not overflow the page
% margins by default, and it is still possible to overwrite the defaults
% using explicit options in \includegraphics[width, height, ...]{}
\setkeys{Gin}{width=\maxwidth,height=\maxheight,keepaspectratio}
\IfFileExists{parskip.sty}{%
\usepackage{parskip}
}{% else
\setlength{\parindent}{0pt}
\setlength{\parskip}{6pt plus 2pt minus 1pt}
}
\setlength{\emergencystretch}{3em}  % prevent overfull lines
\providecommand{\tightlist}{%
  \setlength{\itemsep}{0pt}\setlength{\parskip}{0pt}}
\setcounter{secnumdepth}{0}
% Redefines (sub)paragraphs to behave more like sections
\ifx\paragraph\undefined\else
\let\oldparagraph\paragraph
\renewcommand{\paragraph}[1]{\oldparagraph{#1}\mbox{}}
\fi
\ifx\subparagraph\undefined\else
\let\oldsubparagraph\subparagraph
\renewcommand{\subparagraph}[1]{\oldsubparagraph{#1}\mbox{}}
\fi

%%% Use protect on footnotes to avoid problems with footnotes in titles
\let\rmarkdownfootnote\footnote%
\def\footnote{\protect\rmarkdownfootnote}

%%% Change title format to be more compact
\usepackage{titling}

% Create subtitle command for use in maketitle
\providecommand{\subtitle}[1]{
  \posttitle{
    \begin{center}\large#1\end{center}
    }
}

\setlength{\droptitle}{-2em}

  \title{Moment of Truth: Does Central Limit Theorem Work on an Exponential
Distribution?}
    \pretitle{\vspace{\droptitle}\centering\huge}
  \posttitle{\par}
    \author{Ali Pourkhesalian}
    \preauthor{\centering\large\emph}
  \postauthor{\par}
      \predate{\centering\large\emph}
  \postdate{\par}
    \date{7 July 2019}


\begin{document}
\maketitle

\hypertarget{overview}{%
\subsection{Overview}\label{overview}}

This reports investigates the exponential distribution in R and compares
it with the Central Limit Theorem (CLT).

\#\#Simulations The exponential distribution is simulated in R with
rexp(n, lambda), where lambda is the rate parameter. The mean of
exponential distribution is 1/lambda and the standard deviation is also
1/lambda. In this report, lambda is set to 0.2 for all of the
simulations. The CLT is tested on the distribution of averages of 40
exponentials for a thousand simulations.

\begin{Shaded}
\begin{Highlighting}[]
\NormalTok{lambda<-.}\DecValTok{2}\NormalTok{;n<-}\DecValTok{40}\NormalTok{;mns =}\StringTok{ }\OtherTok{NULL}\NormalTok{;}\ControlFlowTok{for}\NormalTok{ (i }\ControlFlowTok{in} \DecValTok{1} \OperatorTok{:}\StringTok{ }\DecValTok{1000}\NormalTok{) mns =}\StringTok{ }\KeywordTok{c}\NormalTok{(mns, }\KeywordTok{mean}\NormalTok{(}\KeywordTok{rexp}\NormalTok{(n,lambda)))}
\NormalTok{h<-}\KeywordTok{hist}\NormalTok{(mns, }\DataTypeTok{breaks=}\DecValTok{20}\NormalTok{,}\DataTypeTok{xlab=}\StringTok{"Means"}\NormalTok{, }\DataTypeTok{main=}\StringTok{"Means of Exponential Distribution VS CLT"}\NormalTok{) }
\NormalTok{xfit<-}\KeywordTok{seq}\NormalTok{(}\KeywordTok{min}\NormalTok{(mns),}\KeywordTok{max}\NormalTok{(mns),}\DataTypeTok{length=}\DecValTok{80}\NormalTok{) }
\NormalTok{yfit<-}\KeywordTok{dnorm}\NormalTok{(xfit,}\DataTypeTok{mean=}\KeywordTok{mean}\NormalTok{(mns),}\DataTypeTok{sd=}\KeywordTok{sd}\NormalTok{(mns)) }
\NormalTok{yfit <-}\StringTok{ }\NormalTok{yfit}\OperatorTok{*}\KeywordTok{diff}\NormalTok{(h}\OperatorTok{$}\NormalTok{mids[}\DecValTok{1}\OperatorTok{:}\DecValTok{2}\NormalTok{])}\OperatorTok{*}\KeywordTok{length}\NormalTok{(mns) }
\KeywordTok{lines}\NormalTok{(xfit, yfit, }\DataTypeTok{col=}\StringTok{"blue"}\NormalTok{, }\DataTypeTok{lwd=}\DecValTok{2}\NormalTok{)}
\NormalTok{yfitt<-}\KeywordTok{dnorm}\NormalTok{(xfit,}\DataTypeTok{mean=}\KeywordTok{mean}\NormalTok{(}\DecValTok{1}\OperatorTok{/}\NormalTok{lambda),}\DataTypeTok{sd=}\NormalTok{(}\DecValTok{1}\OperatorTok{/}\NormalTok{lambda)}\OperatorTok{/}\KeywordTok{sqrt}\NormalTok{(n)) }
\NormalTok{yfitt <-}\StringTok{ }\NormalTok{yfitt}\OperatorTok{*}\KeywordTok{diff}\NormalTok{(h}\OperatorTok{$}\NormalTok{mids[}\DecValTok{1}\OperatorTok{:}\DecValTok{2}\NormalTok{])}\OperatorTok{*}\KeywordTok{length}\NormalTok{(mns) }
\KeywordTok{lines}\NormalTok{(xfit, yfitt, }\DataTypeTok{col=}\StringTok{"red"}\NormalTok{, }\DataTypeTok{lwd=}\DecValTok{2}\NormalTok{)}
\KeywordTok{abline}\NormalTok{(}\DataTypeTok{v=}\KeywordTok{mean}\NormalTok{(mns), }\DataTypeTok{col=}\StringTok{'blue'}\NormalTok{, }\DataTypeTok{lwd=}\DecValTok{2}\NormalTok{, }\DataTypeTok{lty=}\DecValTok{2}\NormalTok{);}\KeywordTok{abline}\NormalTok{(}\DataTypeTok{v=}\DecValTok{1}\OperatorTok{/}\NormalTok{lambda, }\DataTypeTok{col=}\StringTok{'red'}\NormalTok{, }\DataTypeTok{lwd=}\DecValTok{2}\NormalTok{, }\DataTypeTok{lty=}\DecValTok{2}\NormalTok{)}
\KeywordTok{legend}\NormalTok{(}\DataTypeTok{x=} \StringTok{'topright'}\NormalTok{, }\DataTypeTok{legend=} \KeywordTok{c}\NormalTok{(}\StringTok{"Sample distribution"}\NormalTok{, }\StringTok{"Sample mean"}\NormalTok{, }\StringTok{"Theoritical distribution"}\NormalTok{, }\StringTok{"Theoritical mean"}\NormalTok{), }\DataTypeTok{col =} \KeywordTok{c}\NormalTok{(}\StringTok{'blue'}\NormalTok{,}\StringTok{'blue'}\NormalTok{,}\StringTok{'red'}\NormalTok{,}\StringTok{'red'}\NormalTok{),}\DataTypeTok{text.col =} \KeywordTok{c}\NormalTok{(}\StringTok{'blue'}\NormalTok{,}\StringTok{'blue'}\NormalTok{,}\StringTok{'red'}\NormalTok{,}\StringTok{'red'}\NormalTok{), }\DataTypeTok{lty =} \KeywordTok{c}\NormalTok{(}\DecValTok{1}\NormalTok{,}\DecValTok{2}\NormalTok{,}\DecValTok{1}\NormalTok{,}\DecValTok{2}\NormalTok{),}\DataTypeTok{merge =} \OtherTok{TRUE}\NormalTok{)}
\end{Highlighting}
\end{Shaded}

\includegraphics{c6w4_assignment_p1_files/figure-latex/simulations-1.pdf}

\hypertarget{sample-mean-vs-theoretical-mean}{%
\subsection{Sample Mean VS Theoretical
Mean}\label{sample-mean-vs-theoretical-mean}}

The theoretical mean of the distribution is 5, with a standard deviation
of 0.79, where the sample mean and standard deviation are calculated as
being 5.0152163, and 0.7822395 respectively matching with the
theoritical values very closely.

\hypertarget{sample-variance-vs-theoretical-variance}{%
\subsection{Sample Variance VS Theoretical
Variance}\label{sample-variance-vs-theoretical-variance}}

As it can be seen in the above plot, the distribution of means of an
exponential distribution (the histogram) matches almost perfectly with a
theoritical normal distribution (the red line) and the sample normal
distribution (blue line). The variance of the distribution is calculated
to be 0.6119 comparing to the theoritical value of 0.625, they are quite
close.

\hypertarget{distribution}{%
\subsection{Distribution}\label{distribution}}

The above plot clearly shows that the sample means of an exponential
distribution is distributed normally as expected according to the
Central Limit Theorem (CLT). Hence, it all boils up to: \textbf{CLT
works!}

\hypertarget{appendix}{%
\subsection{Appendix}\label{appendix}}

This document was created by R Markdown. The R Markdown code to generate
the document can be found
\href{https://github.com/pourkhesalian/statistical-inference-course-project}{here}.


\end{document}
